\documentclass[]{article}
\usepackage{lmodern}
\usepackage{amssymb,amsmath}
\usepackage{ifxetex,ifluatex}
\usepackage{fixltx2e} % provides \textsubscript
\ifnum 0\ifxetex 1\fi\ifluatex 1\fi=0 % if pdftex
  \usepackage[T1]{fontenc}
  \usepackage[utf8]{inputenc}
\else % if luatex or xelatex
  \ifxetex
    \usepackage{mathspec}
  \else
    \usepackage{fontspec}
  \fi
  \defaultfontfeatures{Ligatures=TeX,Scale=MatchLowercase}
\fi
% use upquote if available, for straight quotes in verbatim environments
\IfFileExists{upquote.sty}{\usepackage{upquote}}{}
% use microtype if available
\IfFileExists{microtype.sty}{%
\usepackage{microtype}
\UseMicrotypeSet[protrusion]{basicmath} % disable protrusion for tt fonts
}{}
\usepackage[margin=1in]{geometry}
\usepackage{hyperref}
\hypersetup{unicode=true,
            pdftitle={Prophet Customer Forecast},
            pdfborder={0 0 0},
            breaklinks=true}
\urlstyle{same}  % don't use monospace font for urls
\usepackage{color}
\usepackage{fancyvrb}
\newcommand{\VerbBar}{|}
\newcommand{\VERB}{\Verb[commandchars=\\\{\}]}
\DefineVerbatimEnvironment{Highlighting}{Verbatim}{commandchars=\\\{\}}
% Add ',fontsize=\small' for more characters per line
\usepackage{framed}
\definecolor{shadecolor}{RGB}{248,248,248}
\newenvironment{Shaded}{\begin{snugshade}}{\end{snugshade}}
\newcommand{\KeywordTok}[1]{\textcolor[rgb]{0.13,0.29,0.53}{\textbf{{#1}}}}
\newcommand{\DataTypeTok}[1]{\textcolor[rgb]{0.13,0.29,0.53}{{#1}}}
\newcommand{\DecValTok}[1]{\textcolor[rgb]{0.00,0.00,0.81}{{#1}}}
\newcommand{\BaseNTok}[1]{\textcolor[rgb]{0.00,0.00,0.81}{{#1}}}
\newcommand{\FloatTok}[1]{\textcolor[rgb]{0.00,0.00,0.81}{{#1}}}
\newcommand{\ConstantTok}[1]{\textcolor[rgb]{0.00,0.00,0.00}{{#1}}}
\newcommand{\CharTok}[1]{\textcolor[rgb]{0.31,0.60,0.02}{{#1}}}
\newcommand{\SpecialCharTok}[1]{\textcolor[rgb]{0.00,0.00,0.00}{{#1}}}
\newcommand{\StringTok}[1]{\textcolor[rgb]{0.31,0.60,0.02}{{#1}}}
\newcommand{\VerbatimStringTok}[1]{\textcolor[rgb]{0.31,0.60,0.02}{{#1}}}
\newcommand{\SpecialStringTok}[1]{\textcolor[rgb]{0.31,0.60,0.02}{{#1}}}
\newcommand{\ImportTok}[1]{{#1}}
\newcommand{\CommentTok}[1]{\textcolor[rgb]{0.56,0.35,0.01}{\textit{{#1}}}}
\newcommand{\DocumentationTok}[1]{\textcolor[rgb]{0.56,0.35,0.01}{\textbf{\textit{{#1}}}}}
\newcommand{\AnnotationTok}[1]{\textcolor[rgb]{0.56,0.35,0.01}{\textbf{\textit{{#1}}}}}
\newcommand{\CommentVarTok}[1]{\textcolor[rgb]{0.56,0.35,0.01}{\textbf{\textit{{#1}}}}}
\newcommand{\OtherTok}[1]{\textcolor[rgb]{0.56,0.35,0.01}{{#1}}}
\newcommand{\FunctionTok}[1]{\textcolor[rgb]{0.00,0.00,0.00}{{#1}}}
\newcommand{\VariableTok}[1]{\textcolor[rgb]{0.00,0.00,0.00}{{#1}}}
\newcommand{\ControlFlowTok}[1]{\textcolor[rgb]{0.13,0.29,0.53}{\textbf{{#1}}}}
\newcommand{\OperatorTok}[1]{\textcolor[rgb]{0.81,0.36,0.00}{\textbf{{#1}}}}
\newcommand{\BuiltInTok}[1]{{#1}}
\newcommand{\ExtensionTok}[1]{{#1}}
\newcommand{\PreprocessorTok}[1]{\textcolor[rgb]{0.56,0.35,0.01}{\textit{{#1}}}}
\newcommand{\AttributeTok}[1]{\textcolor[rgb]{0.77,0.63,0.00}{{#1}}}
\newcommand{\RegionMarkerTok}[1]{{#1}}
\newcommand{\InformationTok}[1]{\textcolor[rgb]{0.56,0.35,0.01}{\textbf{\textit{{#1}}}}}
\newcommand{\WarningTok}[1]{\textcolor[rgb]{0.56,0.35,0.01}{\textbf{\textit{{#1}}}}}
\newcommand{\AlertTok}[1]{\textcolor[rgb]{0.94,0.16,0.16}{{#1}}}
\newcommand{\ErrorTok}[1]{\textcolor[rgb]{0.64,0.00,0.00}{\textbf{{#1}}}}
\newcommand{\NormalTok}[1]{{#1}}
\usepackage{graphicx,grffile}
\makeatletter
\def\maxwidth{\ifdim\Gin@nat@width>\linewidth\linewidth\else\Gin@nat@width\fi}
\def\maxheight{\ifdim\Gin@nat@height>\textheight\textheight\else\Gin@nat@height\fi}
\makeatother
% Scale images if necessary, so that they will not overflow the page
% margins by default, and it is still possible to overwrite the defaults
% using explicit options in \includegraphics[width, height, ...]{}
\setkeys{Gin}{width=\maxwidth,height=\maxheight,keepaspectratio}
\IfFileExists{parskip.sty}{%
\usepackage{parskip}
}{% else
\setlength{\parindent}{0pt}
\setlength{\parskip}{6pt plus 2pt minus 1pt}
}
\setlength{\emergencystretch}{3em}  % prevent overfull lines
\providecommand{\tightlist}{%
  \setlength{\itemsep}{0pt}\setlength{\parskip}{0pt}}
\setcounter{secnumdepth}{0}
% Redefines (sub)paragraphs to behave more like sections
\ifx\paragraph\undefined\else
\let\oldparagraph\paragraph
\renewcommand{\paragraph}[1]{\oldparagraph{#1}\mbox{}}
\fi
\ifx\subparagraph\undefined\else
\let\oldsubparagraph\subparagraph
\renewcommand{\subparagraph}[1]{\oldsubparagraph{#1}\mbox{}}
\fi

%%% Use protect on footnotes to avoid problems with footnotes in titles
\let\rmarkdownfootnote\footnote%
\def\footnote{\protect\rmarkdownfootnote}

%%% Change title format to be more compact
\usepackage{titling}

% Create subtitle command for use in maketitle
\providecommand{\subtitle}[1]{
  \posttitle{
    \begin{center}\large#1\end{center}
    }
}

\setlength{\droptitle}{-2em}

  \title{Prophet Customer Forecast}
    \pretitle{\vspace{\droptitle}\centering\huge}
  \posttitle{\par}
    \author{}
    \preauthor{}\postauthor{}
    \date{}
    \predate{}\postdate{}
  

\begin{document}
\maketitle

\subsection{Introduction}\label{introduction}

This notebook will outline the steps to produce the prophet forecast,
incorporating historical as well as future data for catalog circulation,
marketing investment, and the J.Jill promotional calendar. The first
step will be loading the required libraries to handle our data in R.

\subsection{Loading the Data}\label{loading-the-data}

There are 4 required files used to build this forecast. They are:

\begin{enumerate}
\def\labelenumi{\arabic{enumi}.}
\tightlist
\item
  Historical customer counts, monthly
\item
  Historical and future investment numbers, monthly
\item
  Historical and future catalog circulation, monthly
\item
  Historical and future promotional calendar
\end{enumerate}

First we will investigate the customer data set.

\subsubsection{Customer Data}\label{customer-data}

\begin{Shaded}
\begin{Highlighting}[]
\CommentTok{# sourced from alteryx workflow: Create Customer Data Set on latest masterXXk.yxdb file}

\NormalTok{proph =}\StringTok{ }\KeywordTok{read.csv}\NormalTok{(}\StringTok{'processed_customers.csv'}\NormalTok{)}
\end{Highlighting}
\end{Shaded}

\subsubsection{Investment Data}\label{investment-data}

\begin{Shaded}
\begin{Highlighting}[]
\CommentTok{# sourced from Maria David, manually processed (no alteryx workflow)}

\NormalTok{inv =}\StringTok{ }\KeywordTok{read.csv}\NormalTok{(}\StringTok{'processed_investments.csv'}\NormalTok{)}
\end{Highlighting}
\end{Shaded}

\subsubsection{Catalog Circulation Data}\label{catalog-circulation-data}

\begin{Shaded}
\begin{Highlighting}[]
\CommentTok{# sourced from alteryx workflow: Prep Catalog History 20180613 on Jim's latest circulation calendar}
\NormalTok{circ =}\StringTok{ }\KeywordTok{read.csv}\NormalTok{(}\StringTok{'processed_circ.csv'}\NormalTok{)}
\end{Highlighting}
\end{Shaded}

\subsubsection{Promotional Calenendar
Data}\label{promotional-calenendar-data}

\begin{Shaded}
\begin{Highlighting}[]
\CommentTok{# sourced from alteryx workflow: Prop Promo File 20180717 on Kaitlyn's latest promo calendar}
\NormalTok{promos =}\StringTok{ }\KeywordTok{read.csv}\NormalTok{(}\StringTok{'processed_promos.csv'}\NormalTok{)}
\end{Highlighting}
\end{Shaded}

Now that we have all of the data prepped and ready for modeling, we can
merge it into one larger file.

\subsubsection{Merge training data}\label{merge-training-data}

\begin{Shaded}
\begin{Highlighting}[]
\NormalTok{regress =}\StringTok{ }\KeywordTok{merge}\NormalTok{(proph, inv, }\DataTypeTok{by=}\KeywordTok{c}\NormalTok{(}\StringTok{'FISCAL_YR'}\NormalTok{, }\StringTok{'FISCAL_MO'}\NormalTok{, }\StringTok{'Segment_Channel'}\NormalTok{), }\DataTypeTok{all.y=}\OtherTok{TRUE}\NormalTok{)}
\NormalTok{regress$ds <-}\StringTok{ }\KeywordTok{as.yearmon}\NormalTok{(}\KeywordTok{paste}\NormalTok{(regress$FISCAL_YR, regress$FISCAL_MO, }\DataTypeTok{sep =} \StringTok{"-"}\NormalTok{))}
\KeywordTok{setnames}\NormalTok{(regress, }\StringTok{'monthly_customers'}\NormalTok{, }\StringTok{'y'}\NormalTok{)}

\NormalTok{regress2 =}\StringTok{ }\KeywordTok{merge}\NormalTok{(regress, promos, }\DataTypeTok{by =} \KeywordTok{c}\NormalTok{(}\StringTok{'FISCAL_YR'}\NormalTok{, }\StringTok{'FISCAL_MO'}\NormalTok{, }\StringTok{'FISCAL_QTR'}\NormalTok{, }\StringTok{'Segment_Channel'}\NormalTok{), }\DataTypeTok{all.y=}\OtherTok{TRUE}\NormalTok{)}
\NormalTok{regress3 =}\StringTok{ }\KeywordTok{merge}\NormalTok{(regress2, circ, }\DataTypeTok{by =} \KeywordTok{c}\NormalTok{(}\StringTok{'FISCAL_YR'}\NormalTok{, }\StringTok{'FISCAL_MO'}\NormalTok{, }\StringTok{'FISCAL_QTR'}\NormalTok{, }\StringTok{'Segment'}\NormalTok{), }\DataTypeTok{all.y=}\OtherTok{TRUE}\NormalTok{)}
\NormalTok{regress3 =}\StringTok{ }\KeywordTok{as.data.table}\NormalTok{(regress3)}

\NormalTok{train =}\StringTok{ }\NormalTok{regress3[FISCAL_YR >=}\StringTok{ }\DecValTok{2016}\NormalTok{]}
\end{Highlighting}
\end{Shaded}

\subsubsection{Define the forecasting
Function}\label{define-the-forecasting-function}

The primary function used to build this forecast is called
\texttt{make\_forecast} defined below. It takes two arguments, a data
frame (df) as well as a threshold (threshold). The threshold is expected
to be a year for which we want to forecast (i.e.~2019 or 2020). If you
want to forecast for 2020, you should set the threshold to 2020,
otherwise use 2019.

The forecast works by instantiating a prophet object (m) and setting a
variety of parmeters relating to holidays, seasonality, and
changepoints. Once the prophet object has been instantiated, a number of
regressor variables can be added (any covariate column in the data frame
argument can be incorporated as a regressor to the model.)

\begin{Shaded}
\begin{Highlighting}[]
\NormalTok{make_forecast <-}\StringTok{ }\NormalTok{function(df, threshold) \{}
  \CommentTok{#threshold = 2019}
  \CommentTok{#df = regress2[Segment_Channel == 'React_Direct1']}
  \NormalTok{df =}\StringTok{ }\KeywordTok{as.data.table}\NormalTok{(df)}
  \NormalTok{df =}\StringTok{ }\NormalTok{df[FISCAL_YR <=}\StringTok{ }\NormalTok{threshold]}
  \NormalTok{m <-}\StringTok{ }\KeywordTok{prophet}\NormalTok{(}\DataTypeTok{seasonality.mode =} \StringTok{'additive'}\NormalTok{, }\DataTypeTok{holidays =} \NormalTok{holidays, }\DataTypeTok{holidays.prior.scale =} \NormalTok{.}\DecValTok{05}\NormalTok{, }\DataTypeTok{seasonality.prior.scale =} \DecValTok{5}\NormalTok{, }\DataTypeTok{changepoint.prior.scale =} \NormalTok{.}\DecValTok{034}\NormalTok{, }\DataTypeTok{weekly.seasonality =} \OtherTok{FALSE}\NormalTok{, }\DataTypeTok{daily.seasonality =} \OtherTok{FALSE}\NormalTok{, }\DataTypeTok{yearly.seasonality =} \OtherTok{FALSE}\NormalTok{)}
  \NormalTok{m <-}\StringTok{ }\KeywordTok{add_regressor}\NormalTok{(m, }\DataTypeTok{name =} \StringTok{'circ_score'}\NormalTok{)}
  \CommentTok{#m <- add_regressor(m, name = 'investment')}
  \CommentTok{#m <- add_regressor(m, name = 'peel_off_depth')}
  \NormalTok{##m <- add_regressor(m, name = 'post_card_depth')}
  \CommentTok{#m <- add_regressor(m, name = 'free_ship_depth')}
  \CommentTok{#m <- add_regressor(m, name = 'flash_sale_depth')}
  \CommentTok{#m <- add_regressor(m, name = 'global_depth')}
  \CommentTok{#m <- add_regressor(m, name = 'fp_entire_depth')}
  \CommentTok{#m <- add_regressor(m, name = 'sp_entire_depth')}
  \NormalTok{m <-}\StringTok{ }\KeywordTok{add_seasonality}\NormalTok{(m, }\DataTypeTok{name=}\StringTok{'yearly'}\NormalTok{, }\DataTypeTok{period=}\FloatTok{365.25}\NormalTok{, }\DataTypeTok{fourier.order=}\DecValTok{10}\NormalTok{, }\DataTypeTok{prior.scale =} \FloatTok{0.05}\NormalTok{)}
  \CommentTok{#m <- add_seasonality(m, name = 'quarterly', period = 365.25/4, fourier.order = 5, prior.scale = 15)}
  \CommentTok{#m <- add_seasonality(m, name = 'monthly', period = 365.25/12, fourier.order = 3, prior.scale = 10)}
  \NormalTok{m <-}\StringTok{ }\KeywordTok{fit.prophet}\NormalTok{(m, df[FISCAL_YR <=}\StringTok{ }\NormalTok{(threshold -}\StringTok{ }\DecValTok{1}\NormalTok{)])}
  \NormalTok{future <-}\StringTok{ }\KeywordTok{make_future_dataframe}\NormalTok{(m, }\DataTypeTok{periods =} \DecValTok{12}\NormalTok{, }\DataTypeTok{freq =} \StringTok{'month'}\NormalTok{)}
  \NormalTok{future$circ_score =}\StringTok{ }\NormalTok{df$circ_score}
  \CommentTok{#future$investment = df$investment  # FIXME this needs to be correctly specified}
  \CommentTok{#future$peel_off_depth = df$peel_off_depth}
  \NormalTok{##future$post_card_depth = df$post_card_depth}
  \CommentTok{#future$free_ship_depth = df$free_ship_depth}
  \CommentTok{#future$flash_sale_depth = df$flash_sale_depth}
  \CommentTok{#future$global_depth = df$global_depth}
  \CommentTok{#future$fp_entire_depth = df$fp_entire_depth}
  \CommentTok{#future$sp_entire_depth = df$sp_entire_depth}
  \NormalTok{fcst <-}\StringTok{ }\KeywordTok{predict}\NormalTok{(m, future)}
  \KeywordTok{prophet_plot_components}\NormalTok{(m, fcst)}
  \KeywordTok{return}\NormalTok{(fcst)}
\NormalTok{\}}
\end{Highlighting}
\end{Shaded}

Note that the function takes in a vector called \texttt{holidays.} We
will need to define this vector to contain information on all of the
J.Jill holidays. We do this manually by referencing the J.Jill fiscal
calendar.

\begin{Shaded}
\begin{Highlighting}[]
\CommentTok{# HOLIDAYS}
\NormalTok{new_years <-}\StringTok{ }\KeywordTok{tibble}\NormalTok{(}
  \DataTypeTok{holiday =} \StringTok{'new_years'}\NormalTok{,}
  \DataTypeTok{ds =} \KeywordTok{as.Date}\NormalTok{(}\KeywordTok{c}\NormalTok{(}\StringTok{'2017-12-01'}\NormalTok{, }\StringTok{'2018-11-01'}\NormalTok{,}
                 \StringTok{'2019-12-01'}\NormalTok{)),}
  \DataTypeTok{lower_window =} \DecValTok{0}\NormalTok{,}
  \DataTypeTok{upper_window =} \DecValTok{1}
\NormalTok{)}

\NormalTok{womens_day <-}\StringTok{ }\KeywordTok{tibble}\NormalTok{(}
  \DataTypeTok{holiday =} \StringTok{'womens_day'}\NormalTok{,}
  \DataTypeTok{ds =} \KeywordTok{as.Date}\NormalTok{(}\KeywordTok{c}\NormalTok{(}\StringTok{'2016-02-01'}\NormalTok{, }\StringTok{'2017-02-01'}\NormalTok{, }\StringTok{'2018-02-01'}\NormalTok{,}
                 \StringTok{'2019-02-01'}\NormalTok{)),}
  \DataTypeTok{lower_window =} \DecValTok{0}\NormalTok{,}
  \DataTypeTok{upper_window =} \DecValTok{1}
\NormalTok{)}

\NormalTok{good_friday <-}\StringTok{ }\KeywordTok{tibble}\NormalTok{(}
  \DataTypeTok{holiday =} \StringTok{'good_friday'}\NormalTok{,}
  \DataTypeTok{ds =} \KeywordTok{as.Date}\NormalTok{(}\KeywordTok{c}\NormalTok{(}\StringTok{'2016-02-01'}\NormalTok{, }\StringTok{'2017-03-01'}\NormalTok{, }\StringTok{'2018-02-01'}\NormalTok{,}
                 \StringTok{'2019-03-01'}\NormalTok{)),}
  \DataTypeTok{lower_window =} \DecValTok{0}\NormalTok{,}
  \DataTypeTok{upper_window =} \DecValTok{1}
\NormalTok{)}

\NormalTok{easter <-}\StringTok{ }\KeywordTok{tibble}\NormalTok{(}
  \DataTypeTok{holiday =} \StringTok{'easter'}\NormalTok{,}
  \DataTypeTok{ds =} \KeywordTok{as.Date}\NormalTok{(}\KeywordTok{c}\NormalTok{(}\StringTok{'2016-02-01'}\NormalTok{, }\StringTok{'2017-03-01'}\NormalTok{, }\StringTok{'2018-02-01'}\NormalTok{,}
                 \StringTok{'2019-03-01'}\NormalTok{)),}
  \DataTypeTok{lower_window =} \DecValTok{0}\NormalTok{,}
  \DataTypeTok{upper_window =} \DecValTok{1}
\NormalTok{)}

\NormalTok{mothers_day <-}\StringTok{ }\KeywordTok{tibble}\NormalTok{(}
  \DataTypeTok{holiday =} \StringTok{'mothers_day'}\NormalTok{,}
  \DataTypeTok{ds =} \KeywordTok{as.Date}\NormalTok{(}\KeywordTok{c}\NormalTok{(}\StringTok{'2016-04-01'}\NormalTok{, }\StringTok{'2017-04-01'}\NormalTok{, }\StringTok{'2018-04-01'}\NormalTok{,}
                 \StringTok{'2019-04-01'}\NormalTok{)),}
  \DataTypeTok{lower_window =} \DecValTok{0}\NormalTok{,}
  \DataTypeTok{upper_window =} \DecValTok{1}
\NormalTok{)}

\NormalTok{memorial <-}\StringTok{ }\KeywordTok{tibble}\NormalTok{(}
  \CommentTok{# on average +17k customers}
  \DataTypeTok{holiday =} \StringTok{'memorial'}\NormalTok{,}
  \DataTypeTok{ds =} \KeywordTok{as.Date}\NormalTok{(}\KeywordTok{c}\NormalTok{(}\StringTok{'2016-05-01'}\NormalTok{, }\StringTok{'2017-05-01'}\NormalTok{, }\StringTok{'2018-04-01'}\NormalTok{,}
                 \StringTok{'2019-04-01'}\NormalTok{)),}
  \DataTypeTok{lower_window =} \DecValTok{0}\NormalTok{,}
  \DataTypeTok{upper_window =} \DecValTok{1}
\NormalTok{)}

\NormalTok{independence <-}\StringTok{ }\KeywordTok{tibble}\NormalTok{(}
  \CommentTok{# on average +5k customers}
  \DataTypeTok{holiday =} \StringTok{'independence'}\NormalTok{,}
  \DataTypeTok{ds =} \KeywordTok{as.Date}\NormalTok{(}\KeywordTok{c}\NormalTok{(}\StringTok{'2016-06-01'}\NormalTok{, }\StringTok{'2017-06-01'}\NormalTok{, }\StringTok{'2018-05-01'}\NormalTok{,}
                 \StringTok{'2019-05-01'}\NormalTok{)),}
  \DataTypeTok{lower_window =} \DecValTok{0}\NormalTok{,}
  \DataTypeTok{upper_window =} \DecValTok{1}
\NormalTok{)}

\NormalTok{labor <-}\StringTok{ }\KeywordTok{tibble}\NormalTok{(}
  \CommentTok{# on average +20k customers}
  \DataTypeTok{holiday =} \StringTok{'labor'}\NormalTok{,}
  \DataTypeTok{ds =} \KeywordTok{as.Date}\NormalTok{(}\KeywordTok{c}\NormalTok{(}\StringTok{'2016-08-01'}\NormalTok{, }\StringTok{'2017-08-01'}\NormalTok{, }\StringTok{'2018-08-01'}\NormalTok{,}
                 \StringTok{'2019-08-01'}\NormalTok{)),}
  \DataTypeTok{lower_window =} \DecValTok{0}\NormalTok{,}
  \DataTypeTok{upper_window =} \DecValTok{1}
\NormalTok{)}

\NormalTok{thanksgiv <-}\StringTok{ }\KeywordTok{tibble}\NormalTok{(}
  \DataTypeTok{holiday =} \StringTok{'thanksgiv'}\NormalTok{,}
  \DataTypeTok{ds =} \KeywordTok{as.Date}\NormalTok{(}\KeywordTok{c}\NormalTok{(}\StringTok{'2016-10-01'}\NormalTok{, }\StringTok{'2017-10-01'}\NormalTok{, }\StringTok{'2018-10-01'}\NormalTok{,}
                 \StringTok{'2019-10-01'}\NormalTok{)),}
  \DataTypeTok{lower_window =} \DecValTok{0}\NormalTok{,}
  \DataTypeTok{upper_window =} \DecValTok{1}
\NormalTok{)}

\NormalTok{black_fri <-}\StringTok{ }\KeywordTok{tibble}\NormalTok{(}
  \DataTypeTok{holiday =} \StringTok{'black_fri'}\NormalTok{,}
  \DataTypeTok{ds =} \KeywordTok{as.Date}\NormalTok{(}\KeywordTok{c}\NormalTok{(}\StringTok{'2016-10-01'}\NormalTok{, }\StringTok{'2017-10-01'}\NormalTok{, }\StringTok{'2018-10-01'}\NormalTok{,}
                 \StringTok{'2019-10-01'}\NormalTok{)),}
  \DataTypeTok{lower_window =} \DecValTok{0}\NormalTok{,}
  \DataTypeTok{upper_window =} \DecValTok{1}
\NormalTok{)}

\NormalTok{cyber_monday <-}\StringTok{ }\KeywordTok{tibble}\NormalTok{(}
  \DataTypeTok{holiday =} \StringTok{'cyber_monday'}\NormalTok{,}
  \DataTypeTok{ds =} \KeywordTok{as.Date}\NormalTok{(}\KeywordTok{c}\NormalTok{(}\StringTok{'2016-11-01'}\NormalTok{, }\StringTok{'2017-11-01'}\NormalTok{, }\StringTok{'2018-10-01'}\NormalTok{,}
                 \StringTok{'2019-11-01'}\NormalTok{)),}
  \DataTypeTok{lower_window =} \DecValTok{0}\NormalTok{,}
  \DataTypeTok{upper_window =} \DecValTok{1}
\NormalTok{)}

\NormalTok{christ_eve <-}\StringTok{ }\KeywordTok{tibble}\NormalTok{(}
  \DataTypeTok{holiday =} \StringTok{'christ_eve'}\NormalTok{,}
  \DataTypeTok{ds =} \KeywordTok{as.Date}\NormalTok{(}\KeywordTok{c}\NormalTok{(}\StringTok{'2016-11-01'}\NormalTok{, }\StringTok{'2017-11-01'}\NormalTok{, }\StringTok{'2018-11-01'}\NormalTok{,}
                 \StringTok{'2019-11-01'}\NormalTok{)),}
  \DataTypeTok{lower_window =} \DecValTok{0}\NormalTok{,}
  \DataTypeTok{upper_window =} \DecValTok{1}
\NormalTok{)}

\NormalTok{christ <-}\StringTok{ }\KeywordTok{tibble}\NormalTok{(}
  \DataTypeTok{holiday =} \StringTok{'christ'}\NormalTok{,}
  \DataTypeTok{ds =} \KeywordTok{as.Date}\NormalTok{(}\KeywordTok{c}\NormalTok{(}\StringTok{'2016-11-01'}\NormalTok{, }\StringTok{'2017-11-01'}\NormalTok{, }\StringTok{'2018-11-01'}\NormalTok{,}
                 \StringTok{'2019-11-01'}\NormalTok{)),}
  \DataTypeTok{lower_window =} \DecValTok{0}\NormalTok{,}
  \DataTypeTok{upper_window =} \DecValTok{1}
\NormalTok{)}

\NormalTok{nye <-}\StringTok{ }\KeywordTok{tibble}\NormalTok{(}
  \DataTypeTok{holiday =} \StringTok{'nye'}\NormalTok{,}
  \DataTypeTok{ds =} \KeywordTok{as.Date}\NormalTok{(}\KeywordTok{c}\NormalTok{(}\StringTok{'2016-11-01'}\NormalTok{, }\StringTok{'2017-12-01'}\NormalTok{, }\StringTok{'2018-11-01'}\NormalTok{,}
                 \StringTok{'2019-11-01'}\NormalTok{)),}
  \DataTypeTok{lower_window =} \DecValTok{0}\NormalTok{,}
  \DataTypeTok{upper_window =} \DecValTok{1}
\NormalTok{)}






\NormalTok{holidays <-}\StringTok{ }\KeywordTok{bind_rows}\NormalTok{(nye, christ, cyber_monday, christ_eve,black_fri, thanksgiv, labor, independence, memorial, mothers_day, easter, good_friday, womens_day, new_years)}
\end{Highlighting}
\end{Shaded}

Now that our function is defined and our holidays are coded, the next
step is to produce a forecast for each of the segment channels in
question. We will use R's built in \texttt{dplyr} package to do this
programmatically.

\subsubsection{Call our function for each segment channel in our
dataframe}\label{call-our-function-for-each-segment-channel-in-our-dataframe}

This block of code will run our make forecast function for 2019 9 times
(note that we are grouping by Segment Channel). This means the forecast
will run once for each segment channel that exists in our data frame.

\begin{Shaded}
\begin{Highlighting}[]
\NormalTok{fcst =}\StringTok{ }\NormalTok{train %>%}\StringTok{  }
\StringTok{  }\KeywordTok{group_by}\NormalTok{(Segment_Channel) %>%}
\StringTok{  }\KeywordTok{do}\NormalTok{(}\KeywordTok{make_forecast}\NormalTok{(., }\DecValTok{2019}\NormalTok{)) %>%}\StringTok{ }
\StringTok{  }\NormalTok{dplyr::}\KeywordTok{select}\NormalTok{(ds, Segment_Channel, yhat)}
\end{Highlighting}
\end{Shaded}

\includegraphics{prophet_writeup_files/figure-latex/unnamed-chunk-8-1.pdf}
\includegraphics{prophet_writeup_files/figure-latex/unnamed-chunk-8-2.pdf}
\includegraphics{prophet_writeup_files/figure-latex/unnamed-chunk-8-3.pdf}
\includegraphics{prophet_writeup_files/figure-latex/unnamed-chunk-8-4.pdf}
\includegraphics{prophet_writeup_files/figure-latex/unnamed-chunk-8-5.pdf}
\includegraphics{prophet_writeup_files/figure-latex/unnamed-chunk-8-6.pdf}
\includegraphics{prophet_writeup_files/figure-latex/unnamed-chunk-8-7.pdf}
\includegraphics{prophet_writeup_files/figure-latex/unnamed-chunk-8-8.pdf}
\includegraphics{prophet_writeup_files/figure-latex/unnamed-chunk-8-9.pdf}

Now we have a forecast at the monthly level. The next step will be to
calculate how this monthly forecast will translate to quarterly and
yearly totals. We will define a function \texttt{calc\_year\_end} to do
this for us. This function will use two files, \texttt{monthly\_pacing}
taken from \texttt{percentages.txt} and \texttt{quarter\_totals} taken
from \texttt{quarter\_totals.txt.} Both of these files were sourced from
alteryx workflow: \texttt{percentages.yxmd}

\begin{Shaded}
\begin{Highlighting}[]
\NormalTok{monthly_pacing =}\StringTok{ }\KeywordTok{fread}\NormalTok{(}\StringTok{"percentages.txt"}\NormalTok{, }\DataTypeTok{sep=} \StringTok{'|'}\NormalTok{)}
\KeywordTok{setnames}\NormalTok{(monthly_pacing, }\StringTok{'Fiscal_Mo'}\NormalTok{, }\StringTok{'FISCAL_MO'}\NormalTok{)}



\NormalTok{quarterly_pacing =}\StringTok{ }\KeywordTok{fread}\NormalTok{(}\StringTok{"quarter_totals.txt"}\NormalTok{, }\DataTypeTok{sep=} \StringTok{'|'}\NormalTok{)}
\NormalTok{quarterly_pacing[, quarter :}\ErrorTok{=}\StringTok{ }\KeywordTok{ifelse}\NormalTok{(FISCAL_MO <=}\StringTok{ }\DecValTok{3}\NormalTok{, }\DecValTok{1}\NormalTok{, }\KeywordTok{ifelse}\NormalTok{(FISCAL_MO <=}\StringTok{ }\DecValTok{6}\NormalTok{, }\DecValTok{2}\NormalTok{, }\KeywordTok{ifelse}\NormalTok{(FISCAL_MO <=}\StringTok{ }\DecValTok{9}\NormalTok{, }\DecValTok{3}\NormalTok{, }\DecValTok{4}\NormalTok{)))]}
\NormalTok{quarterly_pacing[, quarter_sum :}\ErrorTok{=}\StringTok{ }\KeywordTok{sum}\NormalTok{(cust_month), by=}\KeywordTok{c}\NormalTok{(}\StringTok{"quarter"}\NormalTok{, }\StringTok{"Customer_Type"}\NormalTok{)]}
\NormalTok{quarterly_pacing[, table :}\ErrorTok{=}\StringTok{ }\NormalTok{cust_quarter/quarter_sum]}
\NormalTok{quarter_table =}\StringTok{ }\KeywordTok{unique}\NormalTok{(quarterly_pacing[, }\KeywordTok{c}\NormalTok{(}\StringTok{"Customer_Type"}\NormalTok{, }\StringTok{"quarter"}\NormalTok{, }\StringTok{"table"}\NormalTok{, }\StringTok{"quarter_sum"}\NormalTok{)])}
\KeywordTok{setnames}\NormalTok{(quarter_table, }\StringTok{"Customer_Type"}\NormalTok{, }\StringTok{"Segment_Channel"}\NormalTok{)}


\NormalTok{calc_year_end <-}\StringTok{ }\NormalTok{function(cast_dat, year, hist) \{}
  \NormalTok{et =}\StringTok{ }\KeywordTok{as.data.table}\NormalTok{(cast_dat)}
  \NormalTok{et[, FISCAL_MO :}\ErrorTok{=}\StringTok{ }\KeywordTok{month}\NormalTok{(ds)]}
  \NormalTok{et[, quarter :}\ErrorTok{=}\StringTok{ }\KeywordTok{ifelse}\NormalTok{(FISCAL_MO %in%}\StringTok{ }\KeywordTok{c}\NormalTok{(}\DecValTok{1}\NormalTok{,}\DecValTok{2}\NormalTok{,}\DecValTok{3}\NormalTok{), }\DecValTok{1}\NormalTok{, }\KeywordTok{ifelse}\NormalTok{(FISCAL_MO %in%}\StringTok{ }\KeywordTok{c}\NormalTok{(}\DecValTok{4}\NormalTok{,}\DecValTok{5}\NormalTok{,}\DecValTok{6}\NormalTok{), }\DecValTok{2}\NormalTok{, }\KeywordTok{ifelse}\NormalTok{(FISCAL_MO %in%}\StringTok{ }\KeywordTok{c}\NormalTok{(}\DecValTok{7}\NormalTok{,}\DecValTok{8}\NormalTok{,}\DecValTok{9}\NormalTok{), }\DecValTok{3}\NormalTok{,}\DecValTok{4}\NormalTok{)))]}
  \NormalTok{et[, day :}\ErrorTok{=}\StringTok{ }\KeywordTok{format}\NormalTok{(}\KeywordTok{as.Date}\NormalTok{(ds,}\DataTypeTok{format=}\StringTok{"%Y-%m-%d"}\NormalTok{), }\StringTok{"%d"}\NormalTok{)]}
  \NormalTok{et[, FISCAL_YR :}\ErrorTok{=}\StringTok{ }\KeywordTok{year}\NormalTok{(ds)]}
  \NormalTok{et =}\StringTok{ }\NormalTok{et[ day ==}\StringTok{ '01'}\NormalTok{]}
  \KeywordTok{setnames}\NormalTok{(et, }\StringTok{'yhat'}\NormalTok{, }\StringTok{'Used_Forecast'}\NormalTok{)}
  
  
  
  \CommentTok{#wrap up}
  \NormalTok{df2 =}\StringTok{ }\NormalTok{hist[, }\KeywordTok{c}\NormalTok{(}\StringTok{'Segment_Channel'}\NormalTok{, }\StringTok{'FISCAL_YR'}\NormalTok{, }\StringTok{'FISCAL_MO'}\NormalTok{, }\StringTok{'y'}\NormalTok{)]}
  \NormalTok{et2 =}\StringTok{ }\KeywordTok{merge}\NormalTok{(et, df2, }\DataTypeTok{by=}\KeywordTok{c}\NormalTok{(}\StringTok{'Segment_Channel'}\NormalTok{, }\StringTok{'FISCAL_MO'}\NormalTok{, }\StringTok{'FISCAL_YR'}\NormalTok{), }\DataTypeTok{all.x=}\OtherTok{TRUE}\NormalTok{)}
  \NormalTok{et2[, Used_Forecasta :}\ErrorTok{=}\StringTok{ }\KeywordTok{ifelse}\NormalTok{(}\KeywordTok{is.na}\NormalTok{(y), Used_Forecast, y)]}

  
  \NormalTok{year_agg =}\StringTok{ }\KeywordTok{merge}\NormalTok{(et2, monthly_pacing, }\DataTypeTok{by =} \KeywordTok{c}\NormalTok{(}\StringTok{"Segment_Channel"}\NormalTok{, }\StringTok{"FISCAL_MO"}\NormalTok{))}
  \NormalTok{year_agg =}\StringTok{ }\KeywordTok{merge}\NormalTok{(year_agg, quarter_table, }\DataTypeTok{by =} \KeywordTok{c}\NormalTok{(}\StringTok{"quarter"}\NormalTok{, }\StringTok{"Segment_Channel"}\NormalTok{))}
  \NormalTok{year_agg[, Year_Number2 :}\ErrorTok{=}\StringTok{ }\NormalTok{Used_Forecast *}\StringTok{ }\NormalTok{percent]}
  \NormalTok{year_agg[, Year_Number2a :}\ErrorTok{=}\StringTok{ }\NormalTok{Used_Forecasta *}\StringTok{ }\NormalTok{percent]}
  \NormalTok{year_agg[, qnum :}\ErrorTok{=}\StringTok{ }\NormalTok{table*Used_Forecast]}
  \NormalTok{year_agg[, qnuma :}\ErrorTok{=}\StringTok{ }\NormalTok{table *}\StringTok{ }\NormalTok{Used_Forecasta]}

  
  \KeywordTok{return}\NormalTok{(year_agg)}
\NormalTok{\}}
\end{Highlighting}
\end{Shaded}

\subsubsection{Examine results of forecasting
run}\label{examine-results-of-forecasting-run}

\begin{Shaded}
\begin{Highlighting}[]
\NormalTok{final_fcst =}\StringTok{ }\KeywordTok{calc_year_end}\NormalTok{(fcst, }\DecValTok{2019}\NormalTok{, regress3) }
\NormalTok{final_fcst =}\StringTok{ }\NormalTok{final_fcst[FISCAL_YR ==}\StringTok{ }\DecValTok{2019}\NormalTok{]}
\end{Highlighting}
\end{Shaded}

\begin{Shaded}
\begin{Highlighting}[]
\CommentTok{#estimates = c(1, 0.544586951, 0.366562835, 0.337894522)}
\CommentTok{#q_s = final_fcst %>% group_by(quarter, Segment_Channel) %>% summarize(sum(qnuma))}
\CommentTok{#q_s}
\CommentTok{#sum(q_s*estimates) #1921}
\end{Highlighting}
\end{Shaded}

Once we are happy with the output of our forecast, the final step is to
write it to a csv file. The data from this csv file should be pasted
into the \texttt{Prophet\ Forecast\ Data} tab of august\_2020\_v2 excel
workbook. The data and forecast will update automatically after copying
and pasting.

\subsubsection{Write out forecast to csv
file}\label{write-out-forecast-to-csv-file}

\begin{Shaded}
\begin{Highlighting}[]
\KeywordTok{write.csv}\NormalTok{(final_fcst, }\StringTok{'~/Desktop/prophet_aug.csv'}\NormalTok{)}
\end{Highlighting}
\end{Shaded}


\end{document}
